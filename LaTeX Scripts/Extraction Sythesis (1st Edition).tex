\documentclass[12pt,a4paper]{article}
\usepackage[utf8]{inputenc}
\usepackage{amsmath}
\usepackage{amsfonts}
\usepackage{amssymb}
\usepackage{graphicx}
\usepackage{tabularx}
\usepackage{booktabs}
\usepackage{xcolor}
\usepackage{array}
\usepackage{colortbl}
\usepackage{float}

\title{Extraction Synthesis: Audio Synthesis Through Oscillation Isolation\\
\small{1st Edition}}
\author{Digital Down}

\begin{document}

\maketitle

\begin{abstract}
Extraction Synthesis is an audio processing technique that isolates individual oscillations from audio inputs. This technique operates by analyzing the raw audio data at the sample level, identifying oscillations that correspond to musical notes, and reconstructing these isolated oscillations into new audio outputs.
\end{abstract}

\section{Introduction}
Extraction Synthesis begins by converting audio files into a sequence of 16 bit raw audio values. In stereo audio, this results in a list of pairs, where each pair contains the left and right channel values for the sequence of samples. Samples are then scanned for alternating positive and negative values. The sequence of positive and negative values summed together creates a single oscillation. The oscillations are transferred into a new list of sample values that correspond to a musical note.

\section{Methodology}
\begin{center}
Audio Input $\rightarrow$ Conversion to Sample List $\rightarrow$ Processing $\rightarrow$ Conversion to Audio File $\rightarrow$ Audio Output
\end{center}

\subsection{Oscillation Identification}
Extraction Synthesis scans the list of sample values to identify complete oscillations. An oscillation is defined as a sequence of sample values that starts positive, becomes negative, and then returns to positive.\\

Let $S$ be the set of sample values $\{s_1, s_2, ..., s_n\}$

An oscillation $O$ is defined as a subsequence of $S$ where:
\begin{equation*}
O = \{s_i, s_{i+1}, ..., s_j\} \text { where } s_i > 0, s_j > 0, \text { and } \exists k : i < k < j, s_k < 0
\end{equation*}

The algorithm tracks two counters:
\begin{itemize}
    \item Positive Count: The number of consecutive positive samples
    \item Negative Count: The number of consecutive negative samples
\end{itemize}

When a complete oscillation has been scanned, the algorithm checks if an accepted sample sequence has occurred.

\subsection{Note Frequency Matching}
For each identified oscillation, the total sample count (Positive Count + Negative Count) is compared to the expected sample count for each musical note. The expected sample count for a note is calculated as follows:

\begin{equation*}
    \text{Expected Sample Count} = \text{Round}(\frac{\text{Sample Rate}}{\text{Note Frequency}})
\end{equation*}

Where:
\begin{itemize}
    \item Sample Rate = 44100 Samples Per Second
    \item Note Frequency is in Hz
\end{itemize}

For example, for the note A4:
\begin{itemize}
    \item A4 Frequency = 440 Hz
    \item Expected Sample Count = Round(44100 / 440) = 100
\end{itemize}

The algorithm allows for a tolerance of $\pm1$ sample to account for rounding errors and allow more information to pass through.

\subsection{Amplitude Threshold}
To reduce the impact of noise, dithering, and unwanted sounds, the algorithm implements an amplitude threshold. The average absolute value of the samples in an oscillation must exceed this threshold for the oscillation to be accepted.

\begin{equation*}
    \text{Average Absolute Value} = \frac{\sum|s_i|}{n}
\end{equation*}
{\centering
\small\text{($s_i$ are the samples in the oscillation and $n$ is the number of samples)}
\par}

\subsection{Oscillation Extraction and Storage}
When an oscillation meets both the frequency matching and amplitude threshold criteria, it is extracted and added to a list corresponding to its matched note.

\subsection{Note Frequency Tables}
The following tables list all the notes considered in the current implementation, along with their frequencies and expected oscillation counts, divided by octave:

\newcommand{\notetable}[2]{
\begin{table}[H]
\centering
\footnotesize
\begin{tabularx}{\linewidth}{@{}l>{\centering\arraybackslash}Xr@{}}
\toprule
Note & Frequency (Hz) & Expected Oscillation Count \\
\midrule
\multicolumn{3}{c}{\textbf{Octave 1}} \\
\midrule
#1
\bottomrule
\end{tabularx}
\caption{Note Frequencies and Expected Oscillation Counts - Octave #2}
\label{table:note_frequencies_#2}
\end{table}
}

\newcommand{\noterow}[3]{
#1 & #2 & #3 \\
}

\notetable{
\noterow{C1}{32.70}{1348}
\rowcolor{gray!10}
\noterow{C\#1/D$\flat$1}{34.65}{1273}
\noterow{D1}{36.71}{1201}
\rowcolor{gray!10}
\noterow{D\#1/E$\flat$1}{38.89}{1134}
\noterow{E1}{41.20}{1070}
\rowcolor{gray!10}
\noterow{F1}{43.65}{1010}
\noterow{F\#1/G$\flat$1}{46.25}{953}
\rowcolor{gray!10}
\noterow{G1}{49.00}{900}
\noterow{G\#1/A$\flat$1}{51.91}{849}
\rowcolor{gray!10}
\noterow{A1}{55.00}{802}
\noterow{A\#1/B$\flat$1}{58.27}{757}
\rowcolor{gray!10}
\noterow{B1}{61.74}{714}
}{1}

% Combined table for octaves 2, 3, and 4
\begin{table}[H]
\centering
\footnotesize
\begin{tabularx}{\linewidth}{@{}l>{\centering\arraybackslash}Xr@{}}
\toprule
Note & Frequency (Hz) & Expected Oscillation Count \\
\midrule
\multicolumn{3}{c}{\textbf{Octave 2}} \\
\midrule
\noterow{C2}{65.41}{674}
\rowcolor{gray!10}
\noterow{C\#2/D$\flat$2}{69.30}{636}
\noterow{D2}{73.42}{601}
\rowcolor{gray!10}
\noterow{D\#2/E$\flat$2}{77.78}{567}
\noterow{E2}{82.41}{535}
\rowcolor{gray!10}
\noterow{F2}{87.31}{505}
\noterow{F\#2/G$\flat$2}{92.50}{477}
\rowcolor{gray!10}
\noterow{G2}{98.00}{450}
\noterow{G\#2/A$\flat$2}{103.83}{425}
\rowcolor{gray!10}
\noterow{A2}{110.00}{401}
\noterow{A\#2/B$\flat$2}{116.54}{378}
\rowcolor{gray!10}
\noterow{B2}{123.47}{357}
\midrule
\multicolumn{3}{c}{\textbf{Octave 3}} \\
\midrule
\noterow{C3}{130.81}{337}
\rowcolor{gray!10}
\noterow{C\#3/D$\flat$3}{138.59}{318}
\noterow{D3}{146.83}{300}
\rowcolor{gray!10}
\noterow{D\#3/E$\flat$3}{155.56}{283}
\noterow{E3}{164.81}{268}
\rowcolor{gray!10}
\noterow{F3}{174.61}{253}
\noterow{F\#3/G$\flat$3}{185.00}{238}
\rowcolor{gray!10}
\noterow{G3}{196.00}{225}
\noterow{G\#3/A$\flat$3}{207.65}{212}
\rowcolor{gray!10}
\noterow{A3}{220.00}{200}
\noterow{A\#3/B$\flat$3}{233.08}{189}
\rowcolor{gray!10}
\noterow{B3}{246.94}{179}
\midrule
\multicolumn{3}{c}{\textbf{Octave 4}} \\
\midrule
\noterow{C4}{261.63}{169}
\rowcolor{gray!10}
\noterow{C\#4/D$\flat$4}{277.18}{159}
\noterow{D4}{293.66}{150}
\rowcolor{gray!10}
\noterow{D\#4/E$\flat$4}{311.13}{142}
\noterow{E4}{329.63}{134}
\rowcolor{gray!10}
\noterow{F4}{349.23}{126}
\noterow{F\#4/G$\flat$4}{369.99}{119}
\rowcolor{gray!10}
\noterow{G4}{392.00}{113}
\noterow{G\#4/A$\flat$4}{415.30}{106}
\rowcolor{gray!10}
\noterow{A4}{440.00}{100}
\noterow{A\#4/B$\flat$4}{466.16}{95}
\rowcolor{gray!10}
\noterow{B4}{493.88}{89}
\bottomrule
\end{tabularx}
\caption{Note Frequencies and Expected Oscillation Counts - Octaves 2, 3, and 4}
\label{table:note_frequencies_234}
\end{table}

% Combined table for octaves 5 and 6
\begin{table}[H]
\centering
\footnotesize
\begin{tabularx}{\linewidth}{@{}l>{\centering\arraybackslash}Xr@{}}
\toprule
Note & Frequency (Hz) & Expected Oscillation Count \\
\midrule
\multicolumn{3}{c}{\textbf{Octave 5}} \\
\midrule
\noterow{C5}{523.25}{84}
\rowcolor{gray!10}
\noterow{C\#5/D$\flat$5}{554.37}{80}
\noterow{D5}{587.33}{75}
\rowcolor{gray!10}
\noterow{D\#5/E$\flat$5}{622.25}{71}
\noterow{E5}{659.25}{67}
\rowcolor{gray!10}
\noterow{F5}{698.46}{63}
\noterow{F\#5/G$\flat$5}{739.99}{60}
\rowcolor{gray!10}
\noterow{G5}{783.99}{56}
\noterow{G\#5/A$\flat$5}{830.61}{53}
\rowcolor{gray!10}
\noterow{A5}{880.00}{50}
\noterow{A\#5/B$\flat$5}{932.33}{47}
\rowcolor{gray!10}
\noterow{B5}{987.77}{45}
\midrule
\multicolumn{3}{c}{\textbf{Octave 6}} \\
\midrule
\noterow{C6}{1046.50}{42}
\rowcolor{gray!10}
\noterow{C\#6/D$\flat$6}{1108.73}{40}
\noterow{D6}{1174.66}{38}
\rowcolor{gray!10}
\noterow{D\#6/E$\flat$6}{1244.51}{35}
\noterow{E6}{1318.51}{33}
\rowcolor{gray!10}
\noterow{F6}{1396.91}{32}
\noterow{F\#6/G$\flat$6}{1479.98}{30}
\rowcolor{gray!10}
\noterow{G6}{1567.98}{28}
\noterow{G\#6/A$\flat$6}{1661.22}{27}
\rowcolor{gray!10}
\noterow{A6}{1760.00}{25}
\noterow{A\#6/B$\flat$6}{1864.66}{24}
\rowcolor{gray!10}
\noterow{B6}{1975.53}{22}
\bottomrule
\end{tabularx}
\caption{Note Frequencies and Expected Oscillation Counts - Octaves 5 and 6}
\label{table:note_frequencies_56}
\end{table}

\subsection{Reconstruction and Output}
After processing the entire input audio, the extracted oscillations for each note are concatenated and converted back into an audio file. This results in a set of audio files, each containing the isolated oscillations for a specific note.

\section{Conclusion}
The output files are individual notes designed to be used as instruments in an audio production environment, allowing musicians and producers to create custom sounds using the extracted oscillations.
\section*{Further Research}
Further research is currently being done to extract percussive elements from a sound source by isolating formerly rejected oscillations and applying several similar amplitude threshold functions as a separate type of Extraction Synthesis tentatively named “Unfamiliar Extraction Synthesis”. By furthering this research, potentially a more qualitative approach can be developed with more user control over the dynamics of the generated output leading to more desirable audio output.

\end{document}